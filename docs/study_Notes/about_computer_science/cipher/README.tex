\documentclass{ctexart}

\section{信息安全考前简记}

在密码学中, RSA加密算法、乘法密码、仿射密码在解密时都会用到数论中模运算(Modulo Operation)的逆元. 下面整理一下数论中模乘运算的逆元的几种求法.

\subsection{基本概念}

\subsubsection{模运算}

对于正整数 $p$ 和正整数 $a$、$b$,定义如下运算:

\begin{itemize}
\item 取模运算 : $a \% p$ (或 $a \mod p$),表示 $a$ 除以 $p$ 的余数。
\item 模 $p$ 加法:$(a + b) \% p$,其结果是 $a + b$ 算术和除以 $p$ 的余数。
\item 模 $p$ 减法:$(a - b) \% p$,其结果是 $a - b$ 算术差除以 $p$ 的余数。
\item 模 $p$ 乘法:$(a * b) \% p$,其结果是 $a * b$ 算数积除以 $p$ 的余数。
\item 同余式:正整数 $a$、$b$ 对 $p$ 取模,他们的余数相同,记作 $a \equiv b \pmod{p}$。
\end{itemize}

说明:
$n \% p$ 得到结果的正负由被除数 $n$ 决定,与 $p$ 无关。
例如:$7 \% 4 = 3$, $-7 \% 4 = -3$, $-7 \% -4 = -3$.

运算规则
模运算与基本四则运算有些相似,但是除法除外。其规则如下:
\begin{align*}
(a + b) \% p &= (a \% p + b \% p) \% p \\
(a - b) \% p &= (a \% p - b \% p) \% p \\
(a * b) \% p &= (a \% p * b \% p) \% p \\
a ^ b \% p &= ((a \% p) ^ b) \% p \\
\end{align*}

结合律
\begin{align*}
((a + b) \% p + c) \% p &= (a + (b + c) \% p) \% p \\
((a * b) \% p * c) \% p &= (a * (b * c) \% p) \% p \\
\end{align*}

交换律
\begin{align*}
(a + b) \% p &= (b + a) \% p \\
(a * b) \% p &= (b * a) \% p \\
\end{align*}

分配律
\begin{align*}
(a + b) \% p &= (a \% p + b \% p) \% p \\
\end{align*}

重要定理
若 $a \equiv b \pmod{p}$,则对于任意的 $c$,都有$(a + c) \equiv (b + c) \pmod{p}$
若 $a \equiv b \pmod{p}$,则对于任意的 $c$,都有$(a * c) \equiv (b * c) \pmod{p}$
若 $a \equiv b \pmod{p}$,$c \equiv d \pmod{p}$,则有:
\begin{align*}
(a + c) &\equiv (b + d) \pmod{p} \\
(a - c) &\equiv (b - d) \pmod{p} \\
(a * c) &\equiv (b * d) \pmod{p} \\
\end{align*}